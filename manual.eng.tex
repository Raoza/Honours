\documentclass[11pt]{article}
\parindent=0cm
\setlength{\textwidth}{16.0cm}
\topmargin-0.5cm
\textheight22.7cm
\oddsidemargin0cm
\evensidemargin0cm

\begin{document}

\section*{GENREG-manual}

GENREG is a program written by Markus Meringer for generating
connected simple regular graphs.  This version also contains
code added by Gunnar Brinkmann to restrict the generation to
bipartite graphs, and by Brendan McKay to provide output in
the graph6 format and also to have English messages.

To generate the connected $k$-regular graphs on $n$ vertices
call {\bf genreg n k}. To obtain only those graphs with girth 
$\ge t$, you must append $t$ as another parameter, therefore
{\bf genreg n k t}. To control the output you may use the
following options \textbf{after the numerical parameters}:
\begin{itemize}

 \item[\bf -b] only generate bipartite graphs

 \item[\bf -a]
  The \underline{a}djacencylists of the generated graphs are
  written to an ASCII file with suffix \verb|.asc|. There is
  also the girth, a set of generators of the automorphism group
  and its order specified. For example when calling 
  {\bf genreg 4 3 -a} you obtain the file \verb|04_3_3.asc| with the content:
  \begin{verbatim}
  Graph 1:

  1 : 2 3 4
  2 : 1 3 4
  3 : 1 2 4
  4 : 1 2 3
  Taillenweite: 3

  3 : 1 2 4 3
  2 : 1 3 2 4
  2 : 1 4 3 2
  1 : 2 1 3 4
  1 : 3 2 1 4
  1 : 4 2 3 1
  Ordnung: 24 \end{verbatim}
  First the adjacency list is given. In every line there are
  behind the colon the neighbours of the vertex in front
  of the colon. Then we have the girth and afterwards in every
  line an automorphism $\pi$. Before the colon there is 
  $\min\{j \in \underline{n}\ |\ \pi(j)\not=j \}$ and behind
  we have at $i$-th position $\pi(i)$. The given generators
  are representatives of the left cosets of a centralizer chain of the 
  automorphism group (Sims chain).

  If you declare an additional integer $x$, only the first $x$
  graphs are put on the file. In this case the file will
  possibly not contain all the graphs for given $n$, $k$ and $t$.
  Therefore the filename is marked by an additional 
  \verb|-U| (for \underline{u}nfinished),
  e.g. with {\bf genreg 6 3 -a 1} you obtain the file
  \verb|06_3_3-U.asc|.

  If the option {\bf -a} is followed by {\bf stdout}, the output
  is not written to a file, but to {\em stdout}. 
 
 \item[\bf -s]
  The generated graphs are written to a binary file with the suffix
  \verb|.scd|. The following coding (\underline{s}hortcode) is used:

  One after the other all vertices $1,...,n$ are considered. Only
  adjacent vertices with larger number than the considered vertex itself
  are added to the code.
  Thus we have for every edge of the graph exactly
  one entry in the code. For the example above ($n=4$, $k=3$) you
  get 
 
  \hspace{1cm}{\tt 2 3 4 3 4 4} 

  as code. 
  To achieve a further compression of the data, we compare
  the code of the next graph to be constructed with the 
  preceeding and find out, in how many entries at the begining
  the two codes are equal. Instead of writing the common
  pieces twice on the file, we only store its length and
  then the differing entries. Thus as first entry of the file
  we always have zero. The 4-regular graphs on 7 vertices
  have the codes 
  
  \hspace{1cm} {\tt 2 3 4 5 3 4 5 6 7 6 7 6 7 7} and

  \hspace{1cm} {\tt 2 3 4 5 3 4 6 5 7 6 7 6 7 7}

  The file {\bf \verb|07_4_3.scd|} consists of 

  \hspace{1cm}{\tt 0 2 3 4 5 3 4 5 6 7 6 7 6 7 7 6 6 5 7 6 7 6 7 7}

  and has length of 24 byte. This kind of compression gets better for
  big $n$ or $t>3$. The program {\em readscd.c}  contains easy functions
  which are able to read shortcode files. 
%  Meanwhile the code should
%  be absolutely the same as the shortcode of Gunnar Brinkmann's 
%  {\em minibaum}. 

  This option can be followed by an integer or {\bf stdout} if
  only a certain number of graphs are to be stored or you want to use
  {\em stdout}.  

  \item[\bf -g]

  The generated graphs are written to a file in \underline{g}raph6 format.

  This option can be followed by an integer or {\bf stdout} if
  only a certain number of graphs are to be stored or you want to use
  {\em stdout}.

  \item[\bf -e]
  The parameters $n$, $k$ and $t$, the number of generated graphs 
  and the required CPU-time are written to a file with suffix 
  \verb|.erg|. Call of {\bf genreg 21 4 5 -e} produces a file named
  \verb|21_4_5.erg| containing 
  \begin{verbatim}
          GENREG - Generator fuer regulaere Graphen
          21 Knoten, Grad 4, Taillenweite mind. 5
          Erzeugung gestartet...
          8 Graphen erzeugt.
          Laufzeit:20.9s  0.4 Repr./s \end{verbatim}
  As long as the construction is not finished, this file has the
  name \verb|21_4_5-U.erg| (and of course the last two lines are
  missing).

  If option {\bf -e} is not used, GENREG writes this information
  to {\em stderr}.

  \item[-c]
  During the execution of the program the generated graphs are 
  \underline{c}ounted and the number is written to {\em stderr} with
  short intervals.
  If option {\bf -e} is used, GENREG writes this information
  to the \verb|.erg| file.
\end{itemize}

In cases where the computation will be very long, the following
\underline{m}odulo-option is available to split the problem
into several jobs, so it can be run in {\em parallel}
on different machines.
\begin{itemize}
 \item[\bf -m]
  This option must be followed by two integers $i$ and $j$, 
  $1\le i\le j$. It is used to split the problem into $j$ parts.
  Call of {\bf genreg 20 3 -s -m 1 2} causes the program
  to compute only the first of two parts. The code of the graphs
  is written to a file named \verb|20_3_3-1.scd|. When the 
  generation is finished, it is renamed to \verb|20_3_3#1.scd|.
  The remaining graphs can be computed with {\bf genreg 20 3 -s -m 2 2}.

  Remarks:
 
  \begin{itemize}
   \item
    The program is designed to run on UNIX machines.
    To preserve compatibility to other operating systems 
    (OS/2, DOS), filenames are chosen to have
    length at most eight. {\bf genreg 20 3 -s -m 1 2} and
    {\bf genreg 20 3 -s -m 1 3} produce files with the same
    name and differing content.

   \item
    As long as the number of jobs $j$ is not too large, the 
    single parts should take about equal time and produce about
    equally much output.  For very big problems, especially
    with girth $>3$ deviation from this fact may occur.  
    In such cases contact the author 
    (markus@btm2x2mat.uni-bayreuth.de) 
    to decide whether a little tuning will solve the problem
    or it is really hopelessly out of  range.
  \end{itemize}
\end{itemize}  

\section*{Installing GENREG}

These instructions are for UNIX only.  After unpacking the tar
file, edit \verb|makefile| to select a C compiler (\verb|cc| or
\verb|gcc|).  If English messages are required, add \verb|-DENGLISH|
to the definition of \verb|CC|.  (Otherwise, messages will be in
German.)  After editing \verb|makefile|, you can compile the
program using \verb|make| .

\end{document}  
